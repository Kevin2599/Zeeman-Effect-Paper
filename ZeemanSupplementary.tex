\documentclass[]{article}

%opening
\title{Zeeman Effect Lab: Supplementary Answers}
\author{Jack Nelson}

\begin{document}

\maketitle

\begin{enumerate}
	\item Our mean value for the Bohr magneton agreed closely with the theory. Our measured value was $\mu_B = 9.22\times10^{-24}\pm 0.06\times10^{-24} J\cdot T^{-1}$, which is within our measured error of the established value of $9.27\times10^{-24} J\cdot T^{-1}$, and within 0.63\% of the actual value.
	
	\item Yes. The center ring is the 546.1 nm interference ring, which corresponds to the primary spectral emission in the $B=0$ case, as well as the $\Delta M_j = 0$ transitions in the $B\neq0$ cases as well.
	
	\item Nine separate energy levels can be seen, which agrees with theory, as there are 6 separate possible transitions of $\Delta M_j=\pm1$.
	The three $\Delta M_j=0$ transitions have been filtered out by the polarizer.
	
	\item Without the polarizer, there are nine total rings per interference band.
	This agrees with theory, as there are nine separate possible transitions of the $M_j$ quantum number.
	
	\item When viewing the spectra axially to the magnetic field, no change in the interference pattern is observed while changing the polarizer orientation. This fits well with theory, as the $\Delta M_j=\pm1$ spectra are circularly polarized about the axis of the magnetic field, so no orientation of the polarizer will block the light from these transitions.
	
	\item All of these observations are important because they help to confirm the theory behind the Zeeman effect and, even more so, verify fundamental properties of quantum mechanics, such as electron total angular momentum and spin-orbit coupling.
\end{enumerate}

\end{document}
